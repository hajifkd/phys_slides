\documentclass[17pt,ignorenonframetext,]{beamer}
\setbeamertemplate{caption}[numbered]
\setbeamertemplate{caption label separator}{: }
\setbeamercolor{caption name}{fg=normal text.fg}
\beamertemplatenavigationsymbolsempty
\usepackage{lmodern}
\usepackage{amssymb,amsmath}
\usepackage{ifxetex,ifluatex}
\usepackage{fixltx2e} % provides \textsubscript
\ifnum 0\ifxetex 1\fi\ifluatex 1\fi=0 % if pdftex
  \usepackage[T1]{fontenc}
  \usepackage[utf8]{inputenc}
\else % if luatex or xelatex
  \ifxetex
    \usepackage{mathspec}
  \else
    \usepackage{fontspec}
  \fi
  \defaultfontfeatures{Ligatures=TeX,Scale=MatchLowercase}
\fi
\usetheme{metropolis}
% use upquote if available, for straight quotes in verbatim environments
\IfFileExists{upquote.sty}{\usepackage{upquote}}{}
% use microtype if available
\IfFileExists{microtype.sty}{%
\usepackage{microtype}
\UseMicrotypeSet[protrusion]{basicmath} % disable protrusion for tt fonts
}{}
\newif\ifbibliography

% Prevent slide breaks in the middle of a paragraph:
\widowpenalties 1 10000
\raggedbottom

\AtBeginPart{
  \let\insertpartnumber\relax
  \let\partname\relax
  \frame{\partpage}
}
\AtBeginSection{
  \ifbibliography
  \else
    \let\insertsectionnumber\relax
    \let\sectionname\relax
    \frame{\sectionpage}
  \fi
}
\AtBeginSubsection{
  \let\insertsubsectionnumber\relax
  \let\subsectionname\relax
  \frame{\subsectionpage}
}

\setlength{\emergencystretch}{3em}  % prevent overfull lines
\providecommand{\tightlist}{%
  \setlength{\itemsep}{0pt}\setlength{\parskip}{0pt}}
\setcounter{secnumdepth}{0}
\usefonttheme{professionalfonts}\usepackage{txfonts}

\title{Possibility of the Heavy QCD Axion \hfil\tiny arXiv:1504.06084,
1602.07909}
\author{Hajime Fukuda (Kavli IPMU) \newline\scriptsize in collabolation with:
C.W. Chiang, K. Harigaya, M. Ibe and T.T. Yanagida}
\date{\today}

\begin{document}
\frame{\titlepage}

\begin{frame}{Strong CP Problem}

\begin{itemize}
\tightlist
\item
  QCD should break CP symmetry \[
  \theta = \theta_\text{YM} + \arg\det(Y_uY_d)
  \]
\item
  However, the violation looks very small \[
  |\theta| \lesssim 10^{-10}
  \]
\end{itemize}

\end{frame}

\begin{frame}{Peccei-Quinn mechanism \tiny Peccei \& Quinn, 1977}

\begin{block}{\(\text{U}(1)_\text{PQ}\) Symmetry}

\[
q_L\to e^{i\alpha} q_L,\ \ \theta\to\theta+2T(R)\alpha
\]

\begin{itemize}
\tightlist
\item
  \(\text{U}(1)_\text{PQ}\) must be broken at \(f_a\) and a pseudo NG
  Boson \(a\) appears\\
  \hfill \tiny Weinberg 1978, Wilczek 1978
\end{itemize}

\end{block}

\end{frame}

\begin{frame}{Their Original Model}

\begin{itemize}
\tightlist
\item
  The VEVs of 2HDM break EW gauge group and \(\text{U}(1)_\text{PQ}\)
  simultaneously
\item
  It's simple and minimal, but experimentally excluded
\end{itemize}

\end{frame}

\begin{frame}{Which Direction?}

\begin{itemize}
\tightlist
\item
  There are roughly two ways to achieve the PQ mechanism

  \begin{itemize}
  \tightlist
  \item
    Larger \(f_a\), \emph{invisible axion}
  \item
    Heavier \(m_a\), \emph{heavy axion}
  \end{itemize}
\end{itemize}

\end{frame}

\begin{frame}{Axion Mass and Decay Constant}

\begin{block}{Axion Mass}

\[
{m_a}^2 \simeq \frac{m_q \Lambda^3}{{f_a}^2}
\]

\begin{itemize}
\tightlist
\item
  Heavier \(m_a\) with sufficiently large \(f_a\) is hence difficult
\end{itemize}

\end{block}

\end{frame}

\begin{frame}{Larger \(f_a\) isn't Easy, Either}

\begin{itemize}
\tightlist
\item
  Why does no higher dim. op. exist? \[
  \Delta \mathcal L = c\frac{\phi^5}{M_\text{Pl}}
  \] \[
  \Rightarrow\Delta \theta \simeq c\frac{{f_a}^3}{M_\text{Pl}{m_a}^2} \gg 10^{-10},
  \] even for the WW axion
\end{itemize}

\end{frame}

\begin{frame}{Realizing a Heavy Axion}

\begin{itemize}
\tightlist
\item
  (Rubakov, 1997) suggested \emph{a consistent way} to achieve a heavy
  axion
\end{itemize}

\tiny Rubakov 1997; Berezhiani, Gianfagna and Giannotti
2000\newline Hook 2014, HF, Harigaya, Ibe and Yanagida 2015, Albaid,
Dine and Draper 2015\newline (Gherghetta, Nagata and Shifman 2016)

\end{frame}

\begin{frame}{How to Make an Axion Heavier?}

\begin{itemize}
\tightlist
\item
  Another gauge theory is needed
\end{itemize}

\begin{center}
\begin{tikzpicture}
\fill [gray] (10, -1.75) circle (0.25);
\draw [olive, ultra thick]  (7, -1) sin (8,0) cos (9,-1) sin (10,-2) cos (11,-1) sin (12, 0) cos (13,-1);
\draw [teal, ultra thick] (7, 0) sin (8,2) cos (9, 0) sin (10, -2) cos (11, 0) sin (12, 2) cos (13, 0);\pause
\draw [<->, ultra thick] (9.5, -2.3) -- (10.5, -2.3);
\draw (10, -2.7) node {Then how can we align the two $\theta$s?};
\end{tikzpicture}
\end{center}

\end{frame}

\begin{frame}{Copy of SM}

\[
\theta = \theta_\text{YM} + \tikz[baseline=(x.base)] {
  \node(x)[rectangle, fill=teal!20, rounded corners] {$\arg\det(Y_uY_d)$};
  }
\]

\begin{itemize}
\tightlist
\item
  \(\theta'\) must also have Yukawa sector
\item
  Thus, we need a complete copy of SM

  \begin{itemize}
  \tightlist
  \item
    We assume \(\mathbb Z_2\) parity, which is spontaneously broken
  \end{itemize}
\end{itemize}

\end{frame}

\begin{frame}{Our Model}

\usetikzlibrary{automata,positioning}

\begin{center}
\begin{tikzpicture}
\tikzset{block/.style={rectangle, fill=teal!20, rounded corners, align=center}};
\node [block, fill=olive!20] (phi) {$\phi$};
\node [block, above left=2cm and 0.5cm of phi] (s) {$u_R,d_R,Q_L,e_R,L_L$\\$H,N_R$\\{}};
\node [below=0cm of s, anchor=south, blue!40!white] (psi) {\mbox{\boldmath $\psi$}};
\node [block, above right=2cm and 0.5cm of phi] (p) {$u_R',d_R',Q_L',e_R',L_L'$\\$H',N_R'$\\{}};
\node [below=0cm of p, anchor=south, blue!40!white] (psip) {\mbox{\boldmath $\psi'$}};
\draw [->, ultra thick, blue!40!white] (phi) -- (psi);
\draw [->, ultra thick, blue!40!white] (phi) -- (psip);
\node [above=0.3cm of phi, fill=olive!10] {PQ symmetry};
\draw [<->, ultra thick] (s) -- (p);
\node [above=2.3 cm of phi] {$\mathbb Z_2$};
\end{tikzpicture}
\end{center}

\end{frame}

\begin{frame}{Use of spontaneous \(\mathbb Z_2\) breaking}

\begin{itemize}
\tightlist
\item
  Recall \[
  {m_a}^2 \simeq \frac{m_q' \Lambda'^3}{{f_a}^2}
  \]
\item
  We have to increase \(m_q' \propto v'\) and \(\Lambda'\)

  \begin{itemize}
  \tightlist
  \item
    For \(\Lambda'\), we introduce color charged particles and change
    their masses.
  \end{itemize}
\end{itemize}

\end{frame}

\begin{frame}{Cosmological Properties}

\begin{itemize}
\tightlist
\item
  \(\gamma'\) is massless

  \begin{itemize}
  \tightlist
  \item
    The axion must decouple before QCD PT
  \end{itemize}
\item
  Seesaw mechanism in \(\nu'\) is forbidden

  \begin{itemize}
  \tightlist
  \item
    \(\nu'\)s have large Dirac mass
  \item
    No fine-tuning:
    \(\sigma_{\mathbb Z_2} = \sigma_{B'-L'}^2/M_\text{Pl}\)
  \end{itemize}
\end{itemize}

\end{frame}

\begin{frame}{Low Energy Spectrum}

\begin{description}
\tightlist
\item[Axion \(a\)\phantom{hogehogehogehogehoge}]
\(m_a\gtrsim 400\,\text{MeV}\)
\item[Vector like quark \(\psi, \psi'\)]
\(m_\psi=\frac{1}{\sqrt2}gf_a\gtrsim 900\,\text{GeV}\)
\item[Dilaton \(s\)\phantom{hogehogehoge}]
\(m_s=\sqrt{2\lambda}f_a\)\tikz[baseline=(x.base)] {
  \node (x) {$\gtrsim \mathcal O(100)\,\text{GeV}$};\pause
  \node [rectangle, rounded corners, fill=teal!20, minimum width={100pt}] {$\simeq 750\,\text{GeV}$??};
}
\end{description}

\end{frame}

\begin{frame}{Dilaton Decay}

\begin{itemize}
\tightlist
\item
  Obviously, \(\displaystyle\frac{s}{f_a}\partial a\partial a\) is the
  strongest
\item
  Almost no \(s\to2\gamma^{(\prime)}\) decay
\item
  Does it fail? \pause - \textbf{No!}
\end{itemize}

\end{frame}

\begin{frame}{Photons and Photon Jets}

\begin{itemize}
\tightlist
\item
  ECAL can't count the number of \(\gamma\)

  \begin{itemize}
  \tightlist
  \item
    Use ``\(s\to2a\), \(a\to2\text{ collinear $\gamma$}\)'' mode 
  \end{itemize}
\end{itemize}

\begin{center}
\begin{tikzpicture}
\node [circle, white, fill=teal] (s) at (5, 0) {$s$};
\node [circle, white, fill=orange!90!black] (a1) at (3.5, -0.5) {$a$};
\node [circle, white, fill=red!90!black] (g11) at (3.5-1.72, -0.5-1.14) {$\gamma$};
\node [circle, white, fill=red!90!black] (g12) at (3.5-2.05, -0.5+0.195) {$\gamma$};
\node [circle, white, fill=orange!90!black] (a2) at (6.5, 0.5) {$a$};
\node [circle, white, fill=red!90!black] (g21) at (6.5+1.72, 0.5+1.14) {$\gamma$};
\node [circle, white, fill=red!90!black] (g22) at (6.5+2.05, 0.5-0.195) {$\gamma$};
\draw [->, ultra thick] (s) -- (a1);
\draw [->, ultra thick] (s) -- (a2);
\draw [->, thick] (a1) -- (g12);
\draw [->, thick] (a1) -- (g11);
\draw [->, thick] (a2) -- (g22);
\draw [->, thick] (a2) -- (g21);
\end{tikzpicture}
\end{center}

\end{frame}

\begin{frame}{Axion Decay}

\begin{block}{Lagrangian}

\[
\mathcal L_a = N_1\frac{\alpha_s}{8\pi}\frac{a}{f_a} G^{(\prime)}\tilde{G}^{(\prime)}+ 
N_2\frac{\alpha}{8\pi}\frac{a}{f_a} F^{(\prime)}\tilde{F}^{(\prime)} \]

\begin{itemize}
\tightlist
\item
  We need large BR

  \begin{itemize}
  \tightlist
  \item
    \(\text{BR}(s\to4\gamma)=\text{BR}(a\to2\gamma)^{\color{red!90!black}2}\)
  \end{itemize}
\item
  \(a\mbox-G\mbox-G\) coupling looks too strong
\end{itemize}

\end{block}

\end{frame}

\begin{frame}{Is Large BR Possible?}

\begin{block}{Two possibility}

\begin{itemize}
\tightlist
\item
  \(m_a < 3m_\pi\), the threshold of \(a\to2g\)
\item
  Use the mixings with mesons
\end{itemize}

\end{block}

\end{frame}

\begin{frame}{Mixings with Mesons}

\begin{itemize}
\tightlist
\item
  The phase space suppresses \(a\to3\pi\)
\end{itemize}

\begin{center}
\includegraphics[width=8cm]{width.pdf}
\end{center}

\end{frame}

\begin{frame}{Summary}

\begin{itemize}
\tightlist
\item
  The heavy axion is possible
\item
  We need a complete copy of SM
\item
  The diphoton excess can be explained as the dilaton using our model
\end{itemize}

\end{frame}

\end{document}
